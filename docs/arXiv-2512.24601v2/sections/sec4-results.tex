We hypothesize that the effective context window~\citep{hsieh2024rulerwhatsrealcontext, goldman2025reallylongcontextneed, hong2025contextrot} of an LLM cannot be understood independently of the \textit{specific task}. That is, more ``complex'' problems will exhibit degradation at even \textit{shorter} lengths than simpler ones. Because of this, we must characterize tasks in terms of how their complexity \textit{scales with prompt length}.

For example, needle-in-a-haystack (NIAH) problems generally keep `needles' constant as prompt length is scaled. As a result, frontier models can now reliably solve these tasks in RULER~\citep{hsieh2024rulerwhatsrealcontext} in the 1M+ token settings but struggle at far shorter lengths on OOLONG~\citep{bertsch2025oolongevaluatinglongcontext}, a task where the answer depends explicitly on almost every line in the prompt.\footnote{This helps explain the patterns seen in Figure~\ref{fig:rlm-scaling} earlier: \mbox{GPT-5} scales effectively on the S-NIAH task, where the needle size is constant despite longer prompts, but shows faster degradation at increasingly \textit{shorter} context lengths on the \textit{linear}-complexity OOLONG and the \textit{quadratic}-complexity OOLONG-Pairs.}











\subsection{Tasks}
\label{sec4.1-experimental-setup}
We design our evaluation around tasks where we can vary the lengths of the prompts, so we can consider problems whose difficulties scale differently with context length.



\textbf{S-NIAH}. Following the single needle-in-the-haystack task in RULER~\citep{hsieh2024rulerwhatsrealcontext}, we consider a set of 50 single tasks that require finding a specific phrase or number in a large set of unrelated text. Here, the information being sought scales as $O(1)$ with respect to input length.

\textbf{BrowseComp-Plus (1K documents)}~\citep{chen2025browsecompplusfairtransparentevaluation}. A multi-hop question-answering benchmark for DeepResearch~\citep{OpenAI_DeepResearch_2025} questions that requires reasoning over multiple different documents. The benchmark provides a verified offline corpus that is guaranteed to contain gold, evidence, and hard negative documents for each question. Following~\citet{sun2025scalinglonghorizonllmagent}, we use 150 randomly sampled instances as our evaluation set; we provide $1000$ randomly chosen documents as input, in which the gold and evidence documents are guaranteed to exist. We report the percentage of correct answers. The answer to each task requires piecing together information from several documents, making this harder than \textbf{S-NIAH} despite also requiring a constant number of documents.

\textbf{OOLONG}~\citep{bertsch2025oolongevaluatinglongcontext}. A long reasoning benchmark that requires transforming chunks of the input semantically, then aggregating these chunks to form a final answer. We report scoring based on the original paper, which scores numerical answers as $\texttt{score}(\hat{y}) = 0.75^{|y-\hat{y}|}$ and other answers as exact match. We focus specifically on the \texttt{trec\_coarse} split, a set of $50$ tasks over a dataset of questions with semantic labels. Each task requires using nearly all entries of the dataset, and therefore scales linearly in processing complexity relative to the input length.

\textbf{OOLONG-Pairs}. We modify the \texttt{trec\_coarse} split of OOLONG to include $20$ new queries that specifically require aggregating \textit{pairs} of chunks to construct the final answer. We report F1 scores over the answer. Each task requires using nearly all \textit{pairs} of entries of the dataset, and therefore requires processing quadratically-many items relative to the input length. In Appendix~\ref{appx:oolong-pairs}, we provide all queries in this benchmark.


\begin{table*}[ht]
\centering
\caption{Performance comparison of different methods across long-context benchmarks of varying complexity. In \textcolor{gray}{gray} is the average API cost $\pm$ the standard deviation of each method on each task. $^{*}$ indicates runs where a method (sometimes) ran into input context limits. Provider costs were computed under OpenAI for GPT-5 and Fireworks for other models. Non-zero scores are rounded to at least $0.1$.}
\resizebox{\linewidth}{!}{%
\begin{tabular}{l@{\hskip 6pt}c@{\hskip 6pt}c@{\hskip 6pt}c@{\hskip 6pt}c}
\toprule
\textbf{Model} & \textbf{CodeQA} & \textbf{BrowseComp+ (1K)} & \textbf{OOLONG} & \textbf{OOLONG-Pairs} \\
\midrule
\textbf{Task Length $N$ (tokens)} & 23K-4.2M & 6M-11M & 131K & 32K \\
\midrule
\multicolumn{5}{l}{\textbf{GPT-5} {\tiny (with RLM sub-calls to GPT-5-mini)}} \\
\midrule
Base Model     
& \score{24.0$^{*}$}{0.13}{0.07}
& \scoreNA{0.0$^{*}$}
& \score{44.0}{0.14}{0.02}
& \score{0.1}{0.16}{0.10} \\ %

CodeAct (+ BM25)      
& \score{22.0$^{*}$}{0.06}{0.08}
& \score{51.0}{0.71}{1.20}
& \score{38.0}{0.61}{1.06}
& \score{24.7}{0.75}{0.43} \\

CodeAct (+ sub-calls)   
& \score{24.0$^{*}$}{0.06}{0.08}
& \scoreNA{0.0$^{*}$}
& \score{40.0}{0.85}{1.27}
& \score{28.4}{1.11}{0.62} \\

Summary agent     
& \score{58.0}{1.31}{1.46}
& \score{70.5}{0.57}{0.10}
& \score{46.0}{0.13}{0.01}
& \score{0.1}{0.13}{0.09} \\

RLM         
& \score{\textbf{62.0}}{0.11}{0.10}
& \score{\textbf{91.3}}{0.99}{1.22}
& \score{\textbf{56.5}}{0.43}{0.85}
& \score{\textbf{58.0}}{0.33}{0.20} \\

RLM (no sub-calls)   
& \score{58.0}{0.18}{0.56}
& \score{88.0}{0.44}{0.90}
& \score{36.0}{0.37}{0.42}
& \score{43.9}{0.69}{1.16} \\
\midrule
\multicolumn{5}{l}{\textbf{Qwen3-Coder-480B-A35B}} \\
\midrule
Base Model     
& \score{20.0$^{*}$}{0.13}{0.08}
& \scoreNA{0.0$^{*}$}
& \score{36.0}{0.06}{0.00}
& \score{0.1}{0.05}{0.01} \\

CodeAct (+ BM25)      
& \score{24.0$^{*}$}{0.17}{0.08}
& \score{12.7}{0.39}{0.50}
& \score{38.0}{1.51}{1.09}
& \score{0.3}{1.54}{0.35} \\

CodeAct (+ sub-calls)
& \score{26.0$^{*}$}{0.28}{0.30}
& \scoreNA{0.0$^{*}$}
& \score{32.0}{1.83}{1.14}
& \score{0.1}{1.49}{0.46} \\

Summary agent     
& \score{50.0}{1.26}{1.50}
& \score{38.0}{8.98}{2.12}
& \score{44.1}{0.15}{0.01}
& \score{0.31}{0.05}{0.00} \\

RLM         
& \score{56.0}{0.92}{1.23}
& \score{44.7}{0.84}{0.63}
& \score{\textbf{48.0}}{0.61}{0.49}
& \score{\textbf{23.1}}{1.02}{0.52} \\

RLM (no sub-calls)   
& \score{\textbf{66.0}}{0.18}{0.58}
& \score{\textbf{46.0}}{0.82}{0.69}
& \score{43.5}{0.32}{0.13}
& \score{17.3}{1.77}{1.23} \\

\midrule
\multicolumn{5}{l}{\textbf{Qwen3-8B}} \\
\midrule
Base Model     
& \score{4.0$^{*}$}{0.01}{0.00}
& \scoreNA{0.0$^{*}$}
& \scoreNA{0.0$^{*}$}
& \score{0.1}{0.01}{0.00} \\

RLM      
& \score{26.0}{0.04}{0.13}
& \score{2.0}{0.03}{0.06}
& \score{24.0}{0.19}{0.26}
& \score{4.3}{0.05}{0.05} \\

RLM (fine-tuned)   
& \score{\textbf{32.0}}{0.02}{0.02}
& \score{\textbf{14.0}}{0.01}{0.03}
& \score{\textbf{32.0}}{0.04}{0.09}
& \score{\textbf{5.2}}{0.02}{0.02} \\



\bottomrule
\end{tabular}}
\label{tab:main}
\end{table*}


\textbf{LongBench-v2 CodeQA}~\citep{bai2025longbenchv2deeperunderstanding}. A multi-choice code repository understanding split from LongBench-v2 that is challenging for modern frontier models. We report the score as the percentage of correct answers. Each instance requires reasoning over a fixed number of files in a codebase to find the right answer.

\subsection{Methods and Baselines} \label{sec4.2-methods}
We compare \RLM{}s against commonly used task-agnostic inference methods, using two modern LMs, GPT-5 with medium reasoning~\citep{singh2025openaigpt5card} and default sampling parameters, and Qwen3-Coder-480B-A35B~\citep{yang2025qwen3technicalreport} using the sampling parameters described in~\citet{Qwen3-Coder-480B-A35B}. For Qwen3-Coder-480B-A35B, we compute costs based on the compute provider Fireworks~\citep{Qwen3-Coder-Fireworks}. In addition to evaluating the base model on all tasks, we also evaluate the following methods and baselines:

\textbf{CodeAct (+ BM25).} We compare directly to a  CodeAct~\citep{wang2024executablecodeactionselicit} agent that can execute code inside of a ReAct~\citep{yao2023reactsynergizingreasoningacting} loop. Unlike an \RLM, CodeAct does not offload the user prompt to the code environment, and instead provides it directly to the LM. Furthermore, following~\citet{jimenez2024swebenchlanguagemodelsresolve, chen2025browsecompplusfairtransparentevaluation}, we equip this agent with a BM25~\citep{10.1561/1500000019} retriever that indexes the input context for tasks where a retriever is appropriate.

\textbf{CodeAct with sub-calls.} To specifically ablate offloading the context as a variable in the REPL, we evaluate a  CodeAct~\citep{wang2024executablecodeactionselicit} baseline with the ability to invoke sub-LM calls. Compared to \RLM{}s, this method loads the context directly into the model.


\textbf{Summary agent.} Following~\citet{sun2025scalinglonghorizonllmagent,wu2025resumunlockinglonghorizonsearch,yu2025memagentreshapinglongcontextllm}, we consider an iterative agent that compacts the context as it is filled. For example, given a corpus of documents, it will iteratively accumulate the documents and summarize when full. In cases where a single document exceeds the model window, the agent will chunk it to fit within the model context window and invoke the same strategy over these chunks. For the GPT-5 experiments, due to the extremely high cost of applying this strategy to millions of tokens, we use GPT-5-nano for compaction and GPT-5 to provide the final answer.




\textbf{\RLM{} with REPL}. We implement an \RLM{} with a Python REPL environment, which loads a module for querying a sub-LM and uses a system prompt presented in Appendix~\ref{appx3:methods}. For the GPT-5 experiments, we use GPT-5-mini for the recursive LMs and GPT-5 for the root LM, as we found this choice to strike a good balance between the capabilities of RLMs and the cost of the recursive calls. We notate a \RLM{} using a model as \RLM(model), e.g. \RLM(GPT-5).

\textbf{\RLM{} with REPL, no sub-calls}. We provide an ablation of our method, in which the prompt is loaded in a REPL environment without the ability to invoke sub-LM calls.%

{\textbf{Finetuning.} To create \textbf{RLM-Qwen3-8B}, we finetune Qwen3-8B on 1,000 filtered trajectories of Qwen3-Coder-480B-A35B as an \RLM{} with Qwen3-8B sub-calls on LongBenchPro~\citep{chen2026longbenchprorealisticcomprehensive} tasks. We use sampling parameters described in~\citet{Qwen3-8B}, and evaluate the fine-tuned RLM-Qwen3-8B as an \RLM{} on our long context tasks. The key insight for training is that being an effective sub-call model is roughly similar to being a general purpose reasoning model, so we can make the training much more tractable (and seemingly short-horizon) at small scale by focusing on improving the root model's ability to manipulate the REPL and to launch recursive calls. We provide more training details in Appendix~\ref{appx5:training}.}






\section{Results and Discussion} \label{sec4.3-results}
Table~\ref{tab:main} reports our main results. We additionally explore how vanilla frontier model performance and \RLM{} performance degrades as input contexts grow in Figure~\ref{fig:rlm-scaling}.




\begin{figure*}[htb!]
    \centering
    \includegraphics[width=\textwidth]{figures/cost_quartiles_dual_new.png}
    \caption{Cost of \RLM{} and baselines described in \S\ref{sec4.2-methods} plotted at the 25th, 50th, 75th, and 95th percentile of total API cost. We observe comparable or even lower costs for \RLM{}s at the 50th percentile, but sharp increases at the tail end due to potentially long \RLM{} trajectories.}
    \label{fig:quartiles}
\end{figure*}


\textbf{Observation 1: \RLM{}s can scale to the 10M+ token regime and can outperform base LMs and existing task-agnostic agent scaffolds on long context tasks}. Across all tasks, \RLM{}s demonstrate strong performance on prompts well beyond the effective context window of a frontier LM, outperforming base models and common long-context scaffolds by up to $2\times$ the performance while maintaining comparable or cheaper average token costs. Notably, \RLM{}s scale well beyond the base models' context window. For instance, on BrowseComp-Plus (1K), a linearly extrapolated cost for GPT-5-mini ingesting 6-11M input tokens is $\$1.50 - \$2.75$, while \RLM(GPT-5) has an average cost of $\$0.99$ and outperforms both the summarization and retrieval baselines by over $29\%$. 

Furthermore, on tasks where processing costs scale with the input context, \RLM{}s make significant improvements over the base model, even on tasks within the model's context window. On OOLONG, the \RLM{} with GPT-5 and Qwen3-Coder outperform the base model by $28.4\%$ and $33.3\%$ respectively. On OOLONG-Pairs, both GPT-5 and Qwen3-Coder make little progress with F1 scores of $<$$0.1\%$, while the \RLM{} using these models achieve F1 scores of $58.0\%$ and $23.1\%$ respectively, highlighting the emergent capability of \RLM{}s to handle extremely information-dense tasks.


\textbf{Observation 2: The REPL is necessary for handling long inputs, while the recursive sub-calling of \RLM{}s provides strong benefits on information-dense inputs.} A key characteristic of \RLM{}s is offloading the context as a variable in an environment $\mathcal{E}$ that the model can interact with. Even without sub-calling capabilities, our ablation of the \RLM{} is able to scale beyond the context limit of the model and outperform other task-agnostic baselines on most long context settings. On the CodeQA and BrowseComp+ tasks with Qwen3-Coder, this ablation is able to outperform the \RLM{} by $17.9\%$ and $3\%$ respectively. 

On information-dense tasks like OOLONG or OOLONG-Pairs, we observed several cases where recursive LM sub-calling is necessary. In \S\ref{sec4.4-qualitative}, we see \RLM{}(Qwen3-Coder) perform the necessary semantic transformation line-by-line through recursive sub-calls, while the ablation without sub-calls is forced to use keyword heuristics to solve these tasks. Across all information-dense tasks, \RLM{}s outperform the ablation without sub-calling by $10\%$-$59\%$.






\textbf{Observation 3: LM performance degrades as a function of input length and problem complexity, while \RLM{} performance scales better.} The benchmarks {S-NIAH}, {OOLONG}, and {OOLONG-Pairs} contain a fixed number of tasks over contexts with lengths ranging from $2^{13}$ to $2^{18}$. Each benchmark can be loosely categorized by different processing complexity of the input context with respect to length (roughly constant, linear, and quadratic respectively). In Figure~\ref{fig:rlm-scaling}, we directly compare an \RLM{} using GPT-5 to base GPT-5 on each task. We find that GPT-5 performance degrades significantly faster for more complex tasks, while \RLM{} performance degrades at a much slower rate, which aligns with the findings of \citet{goldman2025reallylongcontextneed}. For context lengths beyond $2^{14}$, the \RLM{} consistently outperforms GPT-5.

Furthermore, \RLM{} costs scale proportionally to the complexity of the task, while still remaining in the same order of magnitude of cost as GPT-5 (see Figure~\ref{fig:cost-scaling} in Appendix~\ref{appx1:runtime-cost}). In \S\ref{sec4.4-qualitative}, we explore the choices that the \RLM{} makes that cause these differences in cost. Lastly, in this setting, we also observe that the base LM outperforms \RLM{} in the small input context regime. By construction, a \RLM{} has strictly more representation capacity than an LM. In practice, however, we observe that \RLM{} performance is slightly worse on smaller input lengths, suggesting a tradeoff point between when to use a base LM and when to use an \RLM.

\textbf{Observation 4: The inference cost of \RLM{}s remains comparable to a base LM call but has high variance due to differences in trajectory lengths.} \RLM{}s iteratively interact with their context until they find a suitable answer, leading to large differences in iteration length depending on task complexity. In Figure~\ref{fig:quartiles}, we plot the quartile costs for each method across all experiments in Table~\ref{tab:main} excluding BrowseComp-Plus (1K), as the base models cannot fit any of these tasks in context. For GPT-5, the median \RLM{} run is cheaper than the median base model run, but many outlier \RLM{} runs are significantly more expensive than any base model query. However, compared to the summarization agent which ingests the entire input context, \RLM{}s are up to $3\times$ cheaper while maintaining stronger performance across all tasks because the \RLM{} is able to selectively view context. 



We additionally report runtime numbers of each method in Figures~\ref{fig:runtime-gpt-5},~\ref{fig:runtime-qwen3} in Appendix~\ref{appx1:runtime-cost}, but we note several important caveats. Unlike API costs, these numbers are heavily dependent on implementation details such as the machine used, API request latency, and the asynchrony of LM calls. In our implementation of the baselines and \RLM{}s, all LM calls are blocking / sequential. Nevertheless, similar to costs, we observe a wide range of runtimes, especially for \RLM{}s.

\textbf{Observation 5: \RLM{}s are a model-agnostic inference strategy, but different models exhibit different overall decisions on context management and sub-calling.} While GPT-5 and Qwen3-Coder-480B both exhibit strong performance as \RLM{}s relative to their base model and other baselines, they also exhibit different performance and behavior across all tasks. On BrowseComp-Plus (1k) in particular, \RLM(GPT-5) nearly solves all tasks while \RLM(Qwen3-Coder) struggles to solve half.

We note that the \RLM{} system prompt is fixed for each model across all experiments and is not tuned for any particular benchmark. Between GPT-5 and Qwen3-Coder, the only difference in the prompt is an extra line in the RLM(Qwen3-Coder) prompt warning against using too many sub-calls (see Appendix~\ref{appx3:methods}). We provide an explicit example of this difference in example~\ref{ex:o_212}, where \RLM{}(Qwen3-Coder) launches a sub-call per line in OOLONG while GPT-5 is conservative about sub-querying LMs.

\textbf{Observation 6: Training \RLM{}s on one domain can improve general downstream \RLM{} performance.} Certain behavior in \RLM{} trajectories are common among different domains, such as probing the input and recursively sub-calling on shorter contexts. In Table~\ref{tab:main}, we find that \textbf{RLM-Qwen3-8B}, a Qwen3-8B model that we fine-tuned on \RLM(Qwen3-Coder-480B-A35B) trajectories on a small, \textit{unrelated} set of tasks (LongBenchPro;~\citealt{chen2026longbenchprorealisticcomprehensive}) considerably outperforms the base Qwen3-8B as an \RLM{} by $28.3\%$ on average. Furthermore, its inference costs are much lower due to better decision making and fewer mistakes as an \RLM{}.


\subsection{Emergent Patterns in \RLM{} Trajectories} \label{sec4.4-qualitative}
Even without explicit training, \RLM{}s exhibit interesting context and problem decomposition behavior. We select several examples of snippets from \RLM{} trajectories to understand how they solve long context problems and where they can improve. We discuss particular examples of interesting behavior here, with additional examples in Appendix~\ref{appx2:examples}.

\begin{figure*}[htb!]
    \centering
    \includegraphics[width=0.93\textwidth]{figures/Frame_7.png}
    \caption{\RLM{}s have common patterns in their trajectories when solving tasks. (a) We frequently observed \RLM{}s filtering and interacting with their context through \texttt{regex} code. (b) We found that \RLM{}s can effectively decompose their context through recursive sub-calls (c) On long-output tasks, \RLM{}s are able to solve sub-problems using recursive sub-LM calls and stitch their outputs to form a final output.}
    \label{fig:trajectories}
\end{figure*}
\textbf{Chunking and recursively sub-calling LMs.} \RLM{}s defer essentially unbounded-length reasoning chains to sub-LM calls. The choice of decomposition can greatly affect task performance, especially for information-dense problems. In our experiments, we did not observe complicated partitioning strategies beyond uniform chunking or keyword searches. In Figure~\ref{fig:trajectories}b, \RLM{}(Qwen3-Coder) chunks by newline in a 1000+ line context from OOLONG.


\textbf{Filtering input information using code execution based on model priors.} A key intuition for why the \RLM{} abstraction can maintain strong performance on huge inputs without exploding costs is the LM's ability to filter input context without explicitly seeing it. Furthermore, model priors enable the \RLM{} to narrow the search space and process fewer input tokens. As an example, in Figure~\ref{fig:trajectories}a, we observed \RLM{}(GPT-5) using \texttt{regex} queries to search for chunks containing keywords in the original prompt (e.g. ``festival’’) and phrases it has a prior about (e.g. ``La Union’’). %




\textbf{Passing recursive LM outputs through variables for long output tasks.} \RLM{}s are able to produce essentially unbounded tokens well beyond the limit of the base LM by returning variables in the REPL as output. Through the REPL, the \RLM{} can iteratively construct these variables as a mixture of programmatic and sub-(R)LM output calls. We observed this strategy used heavily in OOLONG-Pairs trajectories, where the \RLM{} stored the output of sub-LM calls over the input in variables and stitched them together to form a final answer (see Figure~\ref{fig:trajectories}c).

